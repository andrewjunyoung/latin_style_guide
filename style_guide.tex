\documentclass{article}

%% Begin package imports %%%%%%%%%%%%%%%%%%%%%%%%%%%%%%%%%%%%%%%%%%%%%%%%%%%%%%%

% Language and font encodings
\usepackage[english]{babel}
% Package for angle quotes etc.
\usepackage[T1]{fontenc}
% Set page size and margins
\usepackage[a4paper,top=3cm,bottom=2cm,left=3cm,right=3cm,marginparwidth=1.75cm]{geometry}
\usepackage[utf8]{inputenc}
\usepackage{hyperref}

%%%%%%%%%%%%%%%%%%%%%%%%%%%%%%%%%%%%%%%%%%%%%%%%%%%%%%%%% End package imports %%
%% Begin custom sequences %%%%%%%%%%%%%%%%%%%%%%%%%%%%%%%%%%%%%%%%%%%%%%%%%%%%%%

\newcommand{\standard}[1]{\quad  \\ \textbf{Standard:} {#1}}

\newcommand{\examples}[1]{
\textbf{Examples:} \begin{itemize}
#1
\end{itemize}
}

\newcommand{\example}[1]{\item \textit{#1}}
\newcommand{\nonexample}[2]{\item \textbf{Not:} \textit{#1} \textbf{Instead:}
  \textit{#2}}
\newcommand{\bothokay}[2]{\item \textit{#1} \textbf{Also:} \textit{#2}}

\newcommand{\nonexamples}[1]{
  \textbf{Non examples:} \begin{itemize}
      #1
  \end{itemize}
}

\newcommand{\explanation}[1]{\textbf{Explanation:} #1}

\newcommand{\langlequote}{\guillemotleft}
\newcommand{\ranglequote}{\guillemotright}

%%%%%%%%%%%%%%%%%%%%%%%%%%%%%%%%%%%%%%%%%%%%%%%%%%%%%%%% End custom sequences %%
%% Begin document contents %%%%%%%%%%%%%%%%%%%%%%%%%%%%%%%%%%%%%%%%%%%%%%%%%%%%%

\title{Young's definitive style guide for the latin alphabet}
\author{Andrew J. Young}

\begin{document}
\maketitle

\begin{abstract}
\end{abstract}

\section{Introduction}
\label{sec:intro}

\section{Upper case}

\subsection{Sentence heads and tails}

\label{s:upper_case:sentence_heads}
\standard{Sentences are to begin with a single capital letter. A capital letter
is defined to be «The upper case form of a single character from the [extended]
Latin alphabet».}

\examples{
  \example{The quick brown fox jumped over the lazy dog.}
  \nonexample{LLamas are an animal native to South America.}{Llamas are an
  animal native to South America.}
  \nonexample{IJsselmeer is the name of the closed off inland bay in the central
  Netherlands.}{Ijsselmeer is the name of the closed off inland bay in the
  central Netherlands.}
}

\explanation{This follows a common convention among all dual case languages to
capitalize the first characters of sentences. We accept this convention, for it
helps guide the eye towards the beginnings of sentences and passages.

All other letters should be lowercase, firstly because they are more readable,
and secondly because this makes the text thereafter more predictable, including
for searching text.

We restrict the use of digraphs because digraphs are language specific and
inconsistent. Some languages treat digraphs as 2 characters for collation, while
others treat the same digraph as 1 character. In loanwords, readers should not
be expected to know rules for digraphs. It is most consistent to treat the latin
alphabet as having exactly 26 letters, and that these are always separate.
}

\label{s:upper_case:sentence_tails}
\standard{All characters in a sentence other than the first character are to be
lower case, unless mentioned in one of the exceptions in this section.}

\subsection{Proper nouns}

\label{s:upper_case:proper_nouns}
\standard{Proper nouns are to be capitalized in all contexts, but only when used
as a noun. In other words, words which take their root from a proper noun should
not be capitalized. Proper nouns normally include \langlequote the \ranglequote
where appropriate.}

\examples{
  \example{I just called Mary Goldberg to let her know that im home.}
  \example{Did you know that she studied at Cambridge University?}
  \nonexample{I thought she studied at the University of Oxford?}{I thought
  she studied at The University Of Oxford?}
  \nonexample{California is a state in the USA.}{California is a state in The
  USA.}
  \nonexample{I believe in god.}{I believe in God.}
  \nonexample{I believe in a God.}{I believe in a god.}
  \nonexample{Did you know that they're German?}{Did you know that they're
  german?}
}

\section{Acronyms}

\label{s:upper_case:proper_nouns}
\standard{Acronyms, including initialisms, are to be written as consecutive
upper case letters with no spaces or delimiters between them.}

\examples{
  \example{The US is a signatory country to NATO.}
  \example{As of 2018, the permanent UN security council members are the China,
  France, Russia, The UK, and The USA.}
}

\standard{Acronyms which are not initialisms may be written in lower case
letters. This should be kept consistent with rules on proper nouns.}

\examples{
  \nonexample{The US is a signatory country to nato.}{The US is a signatory
  country to NATO. \textbf{Or:} The US is a signatory country to Nato.}
  \bothokay{UNICEF is a subsidiary of the UN.}{Unicef is a subsidiary of the
  UN.}
}

\section{Punctuation}

\subsection{Colons}

\standard{Colons are to connect related, otherwise incomplete sentences.}

\standard{Colons are to describe preceding sentences, including with further
deductive or descriptive statements.}

\standard{Colons are used to introduce dialog, both in plays, and in written
passages.}

\standard{Colons are used to indicate omission of text, including letters,
words, or sections [See \ref{s:ellipsis}].}

\examples{
  \example{Kings Cross S:t Pancras}
  \example{D:r Cruz runs an office on the edge of town.}
  \example{The : fox jumps :.}
}

\standard{Colons are used in enumerated lists to separate numbers and following
text.}

\examples{
  \example{We need to \begin{enumerate}
    \item find the bag;
    \item get the money;
    \item be home in time for dinner.
  \end{enumerate}}
}

\subsection{Dash}

\standard{Dashes are used to denote ranges of time and space.}

\examples{
  \example{The French and Indian War (1754 - 1763) was fought in western
  Pennsylvania and along the present US - Canada border.}
  \example{The Tokyo - Osaka express train departs at 3.15 PM.}
}

\standard{Dashes are used to denote relation in certain contexts.}

\examples{
  \example{The US - Canada border spans east - west.}
  \example{The Eagles beat the Hawks 31 - 0.}
}

\standard{Dashes are used to denote subcategorization, including in dates and
locations. When used with numbers it should be used without a space.}

\examples{
  \example{The date is 2018-11-3.}
  \example{He addressed the package to \langlequote China - Beijing - Wangfujing
  \ranglequote.}
  \example{Welcome to the Computing 3-12 course, where we cover advanced
  computer architecture.}
  \example{Answer question 1-a first}
}

\standard{All dashes to be used are en dashes.}

\explanation{}

\subsection{Semicolons}

\standard{Semicolons are to be used for iteration, and in place of iterative
commas.}

\examples{
	\example{The people present were Jamie, a man from New Zealand; John, the
milkman's son; and George, a gaunt kind of man with no friends.}
	\example{Several fast food restaurants can be found within the following
cities: London, England; Paris, France; Dublin, Ireland; Madrid, Spain.}
	\nonexample{Here are three examples of familiar sequences: one, two, three;
a, b, c; first, second, third.}{Here are three examples of familiar sequences:
\langlequote one; two; three \ranglequote ; \langlequote a; b; c \ranglequote ;
\langlequote first; second; third \ranglequote .}
}

\subsection{Question marks}
\label{section:question_marks}

\standard{Opening and closing question marks are to be used to enclose
question sentences.}

\examples{
	\example{She came to me and asked: \langlequote ?`What time is it?
  \ranglequote.}
	\example{\v{C}ingis Xan (?`1162? -- 1227) was the founder of the Mongol Empire.}
}

\standard{Opening and closing question marks follow the same rules as opening
and closing brackets, respectively}

\standard{Question commas are to be used one after another in conjunction.}

\examples{
	\example{?`Is it written in good form? ?`Style? ?`Meaning?}
}

\standard{Question marks take the place of periods when used at the end of
sentences.}

\subsection{Exclamation marks}

Exclamation marks follow the same rules as \ref{section:question_marks}

\examples{
  \example{\langlequote !` Stop ! \ranglequote yelled the guard.}
}

\subsection{Commas}

% TODO: vocative commas?

\standard{Commas are principally used for separation of clauses.}

\standard{When clauses are separated with conjunctions, the comma should appear
before the conjunction.}

\examples{
  \example{Mary walked to the party, but she couldnt walk home.}
  \example{Designer clothes are silly, and i can't afford them anyway.}
  \example{!` Don't push that button, or twelve tons of high explosives will go off
  right under our feet!}
}

\standard{Commas are advised for sentence initiating adverbs such as
\langlequote however \ranglequote; \langlequote in fact \ranglequote; and
\langlequote nevertheless \ranglequote. This includes when these adverbs are
used mid phrase. However, this usage is optional}

\examples{
  \bothokay{Therefore, a comma would be appropriate in this sentence.}{Therefore
  a comma would be appropriate in this sentence.}
  \bothokay{Nevertheless, it is not needed.}{Nevertheless it is not needed.}
}

\standard{When adverbs are used mid phrase, commas are advised.}

\examples{
  \example{In this sentence, furthermore, commas would be called for.}
  \nonexample{This sentence is similar; however, a semicolon is used as
  well.}{This sentence is similar. However, a semicolon is used as well.}
}

\standard{Geographical names do not include commas.}

\examples{
  \example{Take this package to Tallahassee Florida}
}

\standard{Dates do not include commas.}

\examples{
  \example{February 3rd 1922: on that day I discovered the meaning of loss.}
}

\standard{The preferred notation for parenthetical phrases is the parenthesis.
\ref{section:parenthesis}.}

\subsection{Ellipsis}

\standard{Ellipses are to be written as 2 unspaced periods.}

\standard{Ellipses indicate omission of text, including letters, words, or
sections.}

\standard{Ellipses are used to indicate silence.}

\standard{Ellipses are used to denote ranges of time and space [see
\ref{e:dashes}].}

\examples{
  \example{The French and Indian War (1754 .. 1763) was fought in western
  Pennsylvania and along the present US - Canada border.}
  \example{The Tokyo .. Osaka express train departs at 3.15 PM.}
}

\subsection{Periods}

\standard{Periods appear at the end of sentences \ref{def:sentence}, directly
after the final word of the preceding sentence and followed by a single space.
The character immediately after a period should be a capital letter.}

\subsection{Commas}

\standard{Phrase commas appear at the end of phrases \ref{def:phrase},
directly after the final word of the preceding phrase, and followed by a
single space.}

\subsection{Quotation marks}

\standard{By default, languages should use double angle quotes («\dots») to
indicate segments of text with common purpose. This includes for speech,
thoughts, and quotations.}

\examples{
    \example{\langlequote Come closer. \ranglequote he said softly.}
    \example{Hamlet: \langlequote To be or not to be, that is the question.
    \ranglequote}
    \example{At the time i thought \langlequote Gosh, that's a nice dress.
  \ranglequote.}
  \example{In the words of Jean-Paul Sartre: \langlequote Man is condemned to be
  free; because once thrown into the world, he is responsible for everything he
  does. \ranglequote .}
}

%\explanation{There are various clustering styles used in different Latin
%scripts.  Of these the most dominant style is perhaps quotation marks (“” \"\"
%‘’ \'\'), but also brackets ( \lbrack \rbrack () \{\} ⟨⟩ ), single and double
%angle brackets (‹› «»), and, in some style guides, dashes.
%
%The most common global convention is to use quotation marks rather than
%alternatives, which is true across language families and national standards.
%However, standards disagree on where to put commas and how to angle them, with
%standards including »\dots«, „\dots”, ”\dots”, «\dots», and “\dots”. In
%consistency with the usage of brackets, quotation marks should protrude to the
%left for opening quotes, and to the right for closing quotes.
%
%Languages which commonly use \' and \", which dont have distinct opening and
%closing forms, have difficulty with clearly distinguishing beginnings and ends
%of bracketed sections, and often have trouble nesting. This problem is also
%comes about when using dashes, which is also quite rare and nonstandard in most
%languages.
%
%From this, we conclude that we want to use quotation marks with distinct opening
%and closing forms (“” ‘’ ‹› «»). Note that angle quotes are considered to be a
%specific typographical style of quotation mark where the marks are placed in the
%middle of the line.
%
%In order to avoid clashes with the apostrophe, we choose to settle with angle
%quotes over inverted commas. We see in figure
%\ref{fig:inverted_commas_and_apostrophe} that quotation marks clash with
%apostrophes, which are used across Latin scripts and even as necessary letters
%in many alphabets. Compare this with \ref{fig:angle_quotes_and_apostrophe},
%where the two are clearly distinct.
%
%Finally, we decide to use single angle quotes as the default in order to more
%clearly disambiguate them from other forms of punctuation such as inequality
%signs and pointy brackets. This is also in line with popular usage of angle
%quotes.
%}

\subsection{Slashes}

\standard{Slashes are used to indicate alternative choices.}

\examples{
  \example{?` Do you agree Y / N?}
}

\standard{Slashes are used to indicate choices in the case of ambiguity,
particularly when this affects word choice. In some cases, round brackets may be
favored. See also \ref{sec:brackets}.}

\examples{
  \example{The waiter / waitress hasnt come yet.}
  \example{The designation Assyrian / Chaldean / Syriac appeared on american and
  swedish censuses in response to the Syriac naming dispute.}
  \example{If he / she wants to come, then that is completely fine.}
}

\subsection{Nested quotation marks}

\standard{Nested quotation marks should not differ from the outermost quotation
marks. If clarity is warranted between quotation marks, he level of nesting can
be indicated by a subscript number.}

\examples{
  \example{My sister says \langlequote The man told me \langlequote You need
  to leave now \ranglequote \ranglequote.}
  \example{My sister says \langlequote \textsubscript{1} The man told me
  \langlequote \textsubscript{2} You need to leave now \ranglequote
  \textsubscript{2} \ranglequote \textsubscript{1}.}
}

\subsection{Brackets}

\standard{Brackets can be used to indicate choices in the case of ambiguity,
particularly when this affects word choice. See also \ref{sec:slashes}.}

\examples{
  \example{If (s)he wants to come, then that is completely fine.}
  \example{The culprit(s) are still on the run.}
}

\subsubsection{Asides}

\standard{Asides are indicated by square brackets ( \lbrack \rbrack ).}

\examples{
  \example{If the dog falls ill [ which I doubt will happen ] then we can take
  it to the vet.}
  \example{If I slip on ice [ funny though that would be ] its likely to hurt.}
  \example{The lion [ the biggest of the cats ] [ the king of the jungle ] sat
  in its enclosure unstirred by the photographers around it.}
  \example{[ Written on a list ] \langlequote Buy eggs [ done ]. \ranglequote}
}

\subsection{Defunct punctuation}

This section alphabetically lists punctuation which is not recommended for
common use. It is given to provide readers with the ability to look up these
characters should they open this manual. Note that use of any of the below
punctuation is considered as bad practice.

\begin{enumerate}
  \item em dash
  \item hyphen (except as a letter of the latin alphabet)
  \item hyphenminus
  \item quotation marks
  \item quotation marks
\end{enumerate}

\section{Custom notation}

\standard{Custom notation can be introduced, such as in plays, using the
backslash and curly brackets to indicate that text is to be considered a certain
way. These should be defined either before, or alongside their first appearance
within a text.}

\examples{
  \example{Hamlet: \textbackslash aside\{A little more than kin, and less than
  kind.\}}
}

\section{Numerals}

\standard{Enumerated lists should be followed by a colon


\end{document}

%%%%%%%%%%%%%%%%%%%%%%%%%%%%%%%%%%%%%%%%%%%%%%%%%%%%%%% End document contents %%

